\documentclass[twoside,10pt,a4paper]{report}
\usepackage[latin1]{inputenc}
\usepackage[T1]{fontenc}
\usepackage{ae}
\usepackage{fullpage}
\usepackage{url}
\usepackage{ocamldoc}
\usepackage{makeidx}

\usepackage{fancyhdr}
\pagestyle{fancy}
\renewcommand{\headrulewidth}{0.9pt}
\renewcommand{\footrulewidth}{0pt}
\setlength{\headheight}{2.8ex}
\setlength{\footskip}{5ex}
\renewcommand{\chaptermark}[1]{ %
  \markboth{\MakeUppercase{\chaptername}\ \thechapter.\ #1}{}}
\renewcommand{\sectionmark}[1]{}
\setcounter{tocdepth}{2}
\setcounter{secnumdepth}{4}
\usepackage{color}
\definecolor{mygreen}{rgb}{0,0.6,0}

\usepackage[ps2pdf]{hyperref}

\setlength{\parindent}{0em}
\setlength{\parskip}{0.5ex}

%\usepackage{listings}
%\lstloadlanguages{Caml}

\makeindex

\title{Bddapron}


\begin{document}
\maketitle

\tableofcontents

\chapter{Introduction}

\texttt{Bddapron} is a companion library to \texttt{Formula}
(manipulation of finite-type fromula using BDDs) and
\texttt{Apron} (numerical abstract domain library). It extends
\texttt{Formula} with formula of numerical type, and by allowing
decisions on numerical constraints in decision diagrams.

It can be seen more simply as an extension of the APRON abstract domain for
dealing with Boolean and finite-type variables, in addition to numerical
variables.

This library offers the following modules:
\begin{itemize}
\item \texttt{Apronexpr} and \texttt{ApronexprDD} are internal
  modules for numerical expressions; \texttt{apronDD} is an
  internal module for abstract values;
\item \texttt{Bddapronexpr}: formula of finite or numerical types,
  with Boolean or numerical decisions;
\item \texttt{Bddaprondomain}: combined Boolean/numerical abstract
  domain, in which an abstract value is a decision diagram on
  finite type variables, with leafs of type \texttt{'a
    Apron.Abstract1.t}.
\end{itemize}
In addition, two modules (\texttt{BddapronexprE} and
\texttt{BddaprondomainE}) extend resp. \texttt{Bddapronexpr} and
\texttt{Bdddapronomain} by incorporating normalized environments.

\input{ocamldoc.tex}

\end{document}
