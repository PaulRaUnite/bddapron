\documentclass[twoside,10pt,a4paper]{report}
\usepackage[latin1]{inputenc}
\usepackage[T1]{fontenc}
\usepackage{ae}
\usepackage{fullpage}
\usepackage{url}
\usepackage{ocamldoc}
\usepackage{makeidx}

\usepackage{fancyhdr}
\pagestyle{fancy}
\renewcommand{\headrulewidth}{0.9pt}
\renewcommand{\footrulewidth}{0pt}
\setlength{\headheight}{2.8ex}
\setlength{\footskip}{5ex}
\renewcommand{\chaptermark}[1]{ %
  \markboth{\MakeUppercase{\chaptername}\ \thechapter.\ #1}{}}
\renewcommand{\sectionmark}[1]{}
\setcounter{tocdepth}{2}
\setcounter{secnumdepth}{4}
\usepackage{color}
\definecolor{mygreen}{rgb}{0,0.6,0}

\usepackage[ps2pdf]{hyperref}

\setlength{\parindent}{0em}
\setlength{\parskip}{0.5ex}

%\usepackage{listings}
%\lstloadlanguages{Caml}

\makeindex

\title{Formula}


\begin{document}
\maketitle

\tableofcontents

\chapter{Introduction}

This group of modules are ``frontend'' modules for manipulating
BDDs. It offers a higher level interface to the
MLCUDDIDL interface to the C library CUDD, namely
\begin{itemize}
\item Functions for manipulating arrays of BDDs as CPU register, including
  most ALU (Arithmetic and Logical Unit) operations: [Reg].
\item An interface for signed/unsigned integers encoded with BDDs, with
  features like autoresizing: [Int].
\item An interface for enumerated types, with management of labels and types: [Enum];
\item A module for BDD/MTBDD output to file: [Output].
\item A module for BDD/MTBDD variables management: [Env].
\end{itemize}

Next higher-level modules allows to manipulate formula and
expressions mixing discrete types.
\begin{itemize}
\item General finite-type expressions: [Expr0].
\item Boolean formula seen as an (abstract) domain: [Domain0].
\end{itemize}
In addition, two modules ([Expr1] and [Domain1]) extend
resp. [Expr0] and [Domain0] by incorporating normalized
environments.

\input{ocamldoc.tex}

\end{document}
