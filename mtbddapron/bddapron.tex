\documentclass[twoside,10pt,a4paper]{report}
\usepackage[latin1]{inputenc}
\usepackage[T1]{fontenc}
\usepackage{ae}
\usepackage{fullpage}
\usepackage{url}
\usepackage{ocamldoc}
\usepackage{makeidx}

\usepackage{fancyhdr}
\pagestyle{fancy}
\renewcommand{\headrulewidth}{0.9pt}
\renewcommand{\footrulewidth}{0pt}
\setlength{\headheight}{2.8ex}
\setlength{\footskip}{5ex}
\renewcommand{\chaptermark}[1]{ %
  \markboth{\MakeUppercase{\chaptername}\ \thechapter.\ #1}{}}
\renewcommand{\sectionmark}[1]{}
\setcounter{tocdepth}{2}
\setcounter{secnumdepth}{4}
\usepackage{color}
\definecolor{mygreen}{rgb}{0,0.6,0}

\usepackage[ps2pdf]{hyperref}

\setlength{\parindent}{0em}
\setlength{\parskip}{0.5ex}

%\usepackage{listings}
%\lstloadlanguages{Caml}

\makeindex

\title{Bddapron}


\begin{document}
\maketitle

\tableofcontents

\chapter{Introduction}

\section{The \textsc{Bdd} library}

The \textsc{Bdd} library is composed of ``frontend'' modules for
manipulating BDDs. It offers a higher level interface to the
MLCUDDIDL interface to the C library CUDD. The following modules
can be considered as internal.
\begin{itemize}
\item Functions for manipulating arrays of BDDs as CPU register,
  including most ALU (Arithmetic and Logical Unit) operations:
  [Bdd.Reg].
\item An interface for signed/unsigned integers encoded with BDDs,
  with features like autoresizing: [Bdd.Int].
\item An interface for enumerated types, with management of labels
  and types: [Bdd.Enum];
\item A module for BDD/MTBDD output to file: [Bdd.Output].
\end{itemize}

Next higher-level modules allows to manipulate formula and
expressions mixing discrete types.
\begin{itemize}
\item A module for BDD/MTBDD variables management: [Bdd.Env].
\item General finite-type expressions: [Bdd.Expr0].
\item Boolean formula seen as an (abstract) domain: [Bdd.Domain0].
\end{itemize}
In addition, two modules ([Bdd.Expr1] and [Bdd.Domain1]) extend
resp. [Bdd.Expr0] and [Bdd.Domain0] by incorporating normalized
environments.

\section{The \textsc{Bddapron} library}

The \texttt{Bddapron} library includes and extends the
\texttt{Bdd} library (manipulation of finite-type fromula using
BDDs) and \texttt{Apron} (numerical abstract domain library). It
enables formula of numerical type, and allows decisions on
numerical constraints in decision diagrams.

It can be seen more simply as an extension of the APRON abstract
domain for dealing with Boolean and finite-type variables, in
addition to numerical variables.

The following modules can be considered as internal.
\begin{itemize}
\item \texttt{Bddapron.Apronexpr} and
  \texttt{Bddapron.ApronexprDD} are internal modules for numerical
  expressions; \texttt{Bddapron.ApronDD} is an internal module for
  abstract values;
\item \texttt{Bddapron.Env}: environments for BDDAPRON.
\end{itemize}

Next higher-level modules allows to manipulate formula and
expressions mixing discrete types.
\begin{itemize}
\item \texttt{Bddapron.Expr0}: formula of finite or numerical types,
  with Boolean or numerical decisions;
\item \texttt{Bddapron.Domain0}: combined Boolean/numerical abstract
  domain, in which an abstract value is a decision diagram on
  finite type variables, with leafs of type \texttt{'a
    Apron.Abstract1.t}.
\end{itemize}
In addition, two modules (\texttt{Bddapron.Expr1} and
\texttt{Bddapron.Domain1}) extend resp. \texttt{Bddapron.Expr0}
and \texttt{Bdddapron.Domain0} by incorporating normalized
environments.

\input{ocamldoc.tex}

\end{document}
